This is a detailed change log of the book for people interested to see how this book has evolved (to keep in touch with new releases send us an e-mail to subscribe to the newsletter), of course the years are based on the Holocene calendar::
	\begin{itemize}
		\item \textbf{May 12002}
			\begin{itemize}[noitemsep]
				\item Definitions (science, law, theorem, postulate, axiom, corollary, ...)
				\item Operations of addition, subtraction, multiplication, division, power
				\item Concept of Numbers (integers, relative real, fractional, complex algebra, abstract,...)
				\item Domain of definition of variables
				\item Arithmetic polynomials
				\item Absolute value
				\item Binary relations, order relations
				\item Gamma Function Euler, Euler's constant
				\item Electromagnetic wave equation, the wave speed, speed of the light, transported energy
				\item Introduction to Optics
				\item Rule of three, percentages
				\item Net quantities, the cost price of purchase, indices, sale prices, brut prices, net profit, net price, assets
				\item Simple and compound interest, late and early payments
				\item Portfolio management (decision criteria, goodwill, return on investment)
				\item Markowitz Model (utility function, selection criteria)
				\item Vocabulary about Stock Exchange
				\item Plasma Frequency
				\item Probability (universe, events, axioms)
				\item Combinatorial Analysis
				\item Statistics (discrete variables, continuous variable, standard deviation, variance and covariance)
				\item Distribution Functions (discrete uniform law, Bernoulli's law, binomial law, hypergeometric law, multinomial law, Poisson's law, Gauss-Laplace law, Cauchy's law, beta law, gamma law, chi-square law, Student's law)
				\item Estimator, correlation
				\item Matrix of covariance
				\item Statistical adequation tests
			\end{itemize}
		\item \textbf{June 12002}
			\begin{itemize}[noitemsep]
				\item Introduction Set Theory (Zermelo-Fraenkel theory, inclusion, complementarity, intersection, union, product, empty set
				\item Arithmetic, harmonic, geometric, quadratic averages
				\item Univocity
				\item Logarithmic and exponential functions
				\item Golden ratio
				\item Introduction to Fourier series
				\item Coulomb's law, electrostatic field, electrostatic potential
				\item Ampere's Theorem
				\item Biot-Savartlaw, magnetic field, magnetic induction
				\item Maxwell's equations
				\item Dimensions in euclidean and fractal geometry
				\item One-dimensions, two-dimension, three-dimension Geometric shapes
				\item Pythagorean theorem
				\item Physical units systems
				\item The principle of least action and energy conservation
				\item Position, velocity and acceleration
				\item Continuity equation
				\item Bernoulli Equation
				\item The Doppler effect
				\item Restrained Relativity, invariance principle, Lorentz transformations (time, length, velocity addition, increase in mass, Minkowski space-time)
				\item General relativity (world line, punctual event, inertial particles, null cones, space-time vectors spaces and curved planes, metric tensor)
				\item Heisenberg principles of uncertainty, Schrödinger equation, Wave probability density
				\item Introduction to superstrings
			\end{itemize}
		\item \textbf{July 12002}
			\begin{itemize}[noitemsep]
				\item Creating of Table of Contents
				\item Visual representations of functions
				\item 2nd degree polynomials and roots
				\item Operator of vector and scalar fields (gradient, nabla, divergence, curl, laplacian)
				\item Vector Analysis (concept of arrow, set of vectors, scalar multiplication, vector space, linear combinations, generating families, bases of a vector space)
				\item Tensor calculus (Einstein convention, Kronecker symbol, anti-symmetry symbol)
				\item Notations for grouped multiplications (Big Sigma) and sums (Big Pi)
				\item Axioms for set of real number 
				\item Definition of various Inequalities 
				\item Circular and related movements
				\item Inertial forces (Coriolis force)
				\item Drake Equation
			\end{itemize}
		\pagebreak
		\item \textbf{August 12002}
			\begin{itemize}[noitemsep]
				\item New biographies (Hilbert, Riemann, Legendre)
				\item New section "Humor"
				\item Introduction to Topology
				\item Euclid postulates
				\item Gauss plane (complex numbers)
				\item De Moivre's formula
				\item Transformations in the complex plane
				\item Base change and tensor scalar product
				\item Covariant and contravariant components
				\item Relativistic transformations of the momentum
				\item Prefixes of multiples and submultiples of SI units
				\item Quality control (probabilities), efficiency curve, level value of acceptable quality (LQA)
				\item Kepler's laws
				\item Proof of classical gravitational force from Kepler's Law
				\item Proof of area law (second Kepler's Law)
				\item Moment of force
				\item Angular momentum
				\item Theorem of angular momentum
				\item Newton-Poisson Equation
				\item Thomson's, Bohr's and Sommerfeld-Bohr's atom model
				\item Hydrogen spectrum
				\item Assumption of Neutron
				\item Quantum numbers
				\item Pauli exclusion Principles
				\item Classical Analysis of the Schrödinger equation for the ideal rectangular potential
				\item Newton's Laws
				\item Proof of Newton's third law from the principle of least action
				\item Release speed (for rockets or others)
				\item Definition of "Scientism"
				\item Variation of the gravitational acceleration in and out of a homogeneous spherical body
				\item The principle of least action and quantum physics (semi-classical limit)
				\item Ballistics (maximum range, safety parabola )
				\item Introduction to nuclear physics
				\item Atomic number, mass number
				\item Radioactivity, activity, decay chain, isotopes, isotones, nuclide dating
				\item Atomic mass system (UMA)
				\item Mass defaults
				\item Fusion and fission
				\item Alpha and beta (minus and plus) disintegration
				\item Electron capture
				\item Gamma emission
			\end{itemize}
		\pagebreak
		\item \textbf{September 12002}	
			\begin{itemize}[noitemsep]
				\item New Biographies (Dalton, Boltzmann, De Broglie, etc.)
				\item Principle of good order
				\item Archimedean Property
				\item Induction Principle
				\item Divisibility (Euclidean division)
				\item Congruent numbers
				\item Proof by nine
				\item Numbers basis
				\item Definition utility of thousands separators
				\item Priorities of parenthesis, brackets, braces and arithmetic operators
				\item Proof Theory (introduction)
				\item Definition of terms, formulas and demonstrations
				\item Definition of languages, symbols, relations and functions
				\item Definition of periodic, compound, basic, rational, fractional, irrational, algebraic and transcendental numbers
				\item Generalization of elementary algebra
				\item Dimensions of a vector space, extension of a free family, rank of a finite family
				\item Vector Hyperplane, direct sums
				\item Almost rigorous definition of the concepts of line, surface (plane) and volume
				\item Straight line intersection, half-lines, segments, aliquot part
				\item Continuity axiom of the line and of the plane
				\item Translation and rotation of plan
				\item Angles, units, measurements, sides of the angle, sharp edges, flat angles, equal angles ,straight angles, acute angles, obtuse angles, supplementary angles, complementary angles, 
				\item Perpendicular lines, bisecting angle
				\item Definition of work and energy: kinetic and potential energy theorem (motor work and resisting work)
				\item Concept of conservative vector field
				\item Conservation of energy and momentum
				\item Mass center theorem
				\item Relativistic force transformation
				\item Relativistic transformations of electric and magnetic fields
				\item Chandrasekhar limit weights (collapse limit of white dwarfs)
				\item Definitions of optics, generalization of the law of refraction
				\item Broglie normalization condition, linked and non-linked states
				\item Harmonic oscillator
				\item Quantum chemistry and molecular vibrations
			\end{itemize}
		\item \textbf{October 12002}
			\begin{itemize}[noitemsep]
				\item New biographies (Cauchy, Neumann, Bessel, Archimedes)
				\item New section on "Theoretical Computing"
				\item Greatest common divisor, least common multiple
				\item Rule signs (...)
				\item Proof of the irrationality of a number
				\item Introduction to arithmetic, harmonic and geometric sequences
				\item Limits and continuity of functions
				\item Definition of affine spaces
				\item Introduction to the Euclidean tensors and their properties
				\item Triangles and properties of triangles
				\item Definition of thermodynamic systems
				\item Definition of the reduced mass
				\item Bessel's functions
				\item Definition of the moment of inertia
				\item Introduction to Quantum Field Theory
				\item Introduction to radiation protection
				\item Proof of Bethe-Bloch formula
				\item Tunnel effect in quantum physics
				\item Introduction Dirac's formalism
				\item Heron and Archimedes algorithm's
				\item Introduction to fractal sets
				\item Introduction to game theory (cooperative games, earnings, payoff matrix, extensive forms, Pareto optimums, Nash equilibrium, evolutionary games)
			\end{itemize}
		\item \textbf{November 12002}
			\begin{itemize}[noitemsep]
			\item Fundamental theorem of arithmetic
			\item Introduction to cryptography (RSA, DES, MD5, SHA-1)
			\item Euler phi function
			\item Small Fermat theorem
			\item Introduction to topological dimensions and scaling 
			\item Cosine directors
		\end{itemize}
		\item \textbf{December 12002}
			\begin{itemize}[noitemsep]
			\item New Biographies (Nash, Cartan, Lucas, Lie)
			\item Detailed development of the "Integer" function
			\item Superposition of linear quantum states (quantum coherence) 
			\item Definition of Lipschitz functions and contracting functions 
			\item Definition of a convergent Cauchy sequence 
			\item Fixed point theorem (used in fractals, Newton methods and many others)
			\item Definitions of reality, the problem of the theory and trial on reality
			\item Definition of Euclidean space and Euclidean affine space
			\item Definition of property concepts in the field of chemistry
		\end{itemize}
		\item \textbf{January 12003}
		\begin{itemize}[noitemsep]
			\item Pascal's Theorem
			\item Buoyancy (Archimedes' principle)
			\item Simple Introduction to different symmetries in physics (temporal, spatial)
			\item Simple Introduction to different transformations in the plane (translation, scaling, reflection, isometry, rotation)
			\item Definition of an inverse and composed function/application
			\item Trigonometry (introduction, remarkable relations/identities, spherical trigonometry)
			\item Signature of a vector space
			\item Schmidt orthogonalization methods, base changes, Fourier associated spaces
			\item Everything (almost) on the plane and spherical trigonometry
			\item Keplerian orbital trajectories
			\item Introduction to the neoclassical monetary model (Say/Walras laws, homogeneity assumption)
			\item Boolean algebra (simple properties and theorems)
			\item Redesigned of quantum physics section (order of the subjects)
			\item Proof of evolutionary Schrödinger  equation
			\item Proof of the relativistic evolutionary Schrödinger equation
			\item Introduction to Antimatter theory
		\end{itemize}
	\item \textbf{February 12003}
			\begin{itemize}[noitemsep]
			\item Proof of gradient, divergence, rotational and Laplacian in cartesian, polary, cylindrical and spherical coordinates
			\item Complete mathematical developments of speed and acceleration expressions in cartesian, polar, cylindrical and spherical coordinates 
			\item Proof  of the relativistic invariance of the electric charge (charge conservation equation) 
			\item Proof of the existence of anti-particle  with opposite charge 
			\item Introduction to Gauge Theory (quadripotentiel Lorenz gauge, Coulomb gauge, Alembertian) 
			\item Introduction to Lagrangian and Hamiltonian formalism (generalized coordinates, configuration spaces, Euler-Lagrange equation, canonical formalism, Legendre transformation, Poisson brackets) 
			\item Rigorous definition of the principle of least action
			\item Definition of Tensor spaces
			\end{itemize}
	\item \textbf{March 12003}
		\begin{itemize}[noitemsep]
			\item New Biographies (Lorentz, Minkowski Hermann, Ricci-Curbastro, Levi-Civita)
			\item Definition of Cartesian product and extension of the scope of application of Cardinals
			\item Proof of Cauchy-Schwarz inequality
			\item Proof of the triangle inequality 
			\item Definitions of the vector product and mixed product 
			\item Proof of condensed form the sum of the first $n$ integers 
			\item Proof of the validity of the integration by parts 
			\item Definition of the algebraic vectorial structure and of an algebra 
			\item Definition of a homomorphism, isomorphism, endomorphism, automorphism
		\end{itemize}
	\item \textbf{April 12003}
		\begin{itemize}[noitemsep]
			\item New Biographies (Göpper-Meyer, Yukawa, Nöther, Cournot)
			\item Cournot Theory equilibrium (competition)
			\item Wilson's model (inventory management)
			\item Mathematics of phasers
			\item Relativistic model of the Sommerfeld's atom
			\item Analytical resolution of the Schrödinger equation
			\item Heisenberg's principles of quantum  uncertainty
			\item Lagrangian formalism of quantum physics fields
			\item Improvement of website forum (add mathematical symbols and external files)
			\item 10 new links to interesting web pages (associations + math stuff)
		\end{itemize}
	\item \textbf{May 12003}
		\begin{itemize}[noitemsep]
			\item New Biographies (Bell, Ramanujan, Landau)
			\item Proof of the precession of the perihelion of coupled orbits of stars or electric charges
			\item Definition and developments related to the Virial theorem
			\item Calculation of potential energy of a material sphere (internal temperature of Stars)
			\item Definition of prime numbers and proof that they are in infinite in number
			\item Definition of a fully enclosed ring
			\item Proof that a rational number is an algebraic number if and only if it is a relative integer.
			\item Definition of a multi-linear application/function (or morphism of vector space)
			\item Definition of a partition of a set and an equivalence class.
			\item Descriptions of set operations of absorption and idempotence.
			\item Definitions and examples of sagittals diagrams.
			\item Definition of a magma and a monoid
			\item Pseudo-proofs of algebraic structures of fundamental sets of arithmetic
			\item Development of the theory of angular momentum in wave quantum physics
			\item Definition of a Diophantine equation and sets of Fermat's Last Theorem
		\end{itemize}
	\item \textbf{October 12003}
		\begin{itemize}[noitemsep]
			\item New Biographies (Abel, Banach, Boole, Bose, Brouwer, Clausius, Cayley, Curie,  Connes, Dirichlet, Frege, Gibbs, Picard, Erdos, Grothendieck, Hamilton, Hausdorf,  Heaviside, Helmholtz, Hermite, Hoyle, Jacobi, Klein, Kronecker, Langevin, Lee, Lobachevsky , Möbius, Monge, Fish, Schwartz, Shannon, Thom, van der Waals, Vieta, Weinberg, Witten, Gamow, Sturm, Liouville,Clairaut, Teller)
			\item Definition of punctual spaces in classical mechanics, of writing conventions and changes in referential
			\item Mathematical theory of projective perspective with vanishing points and definition of the projective and isometric perspectives
			\item Simplistic definition of the concept of "derivative" in functional analysis.
			\item Proof of some common derivatives (polynomials, composite functions, inverse functions, cosine, sine, arc sine, arc cosine, quotient of two functions, etc.).
			\item Numerical Methods: mathematical explanation of the complexity of an algorithm and  elementary optimization research
			\item Method of calculating the number $e$
			\item Resolution Method of $n$ linear equations systems with $n$ unknown using the pivot method
			\item Search of function roots by the methods of the proportional parts, of the bisection method, of the secant methods (regula falsi) and Newton's method
			\item Calculations of areas and integrals using the Riemann sums method
			\item Presentation of the Monte Carlo calculation for integrals, or Pi
			\item Mathematical development of simplex algorithm used in the operations research (linear programming)
			\item Contraction of indices in tensor calculus
			\item Definition and detailed properties of some special tensor (symmetric tensor, antisymmetric tensor, fundamental tensor, etc.)
			\item Curvilinear coordinates (determination of the metric and linear element of a spherical and cartesian space and of the plane in polar coordinates)
Christoffel symbols (first and second type).
			\item Proof of the Cantor-Bernstein theorem
			\item Determination of the free generalized Lagrangian in General Relativity
		\end{itemize}
	\item \textbf{December 12003}
		\begin{itemize}[noitemsep]
			\item New Biographies (Kirchhoff biographers, Markowitz, Cox Merton, Scholes, Sharpe, Ferdinan von Lindemann, Bachelier, Stefan)
			\item Proof of one of the Stirling formulas
			\item Detailed calculation of a simple model of a Star surface temperature
			\item Detailed calculations of temporal properties of the securities values
			\item Proof of Taylor series and Maclaurin (limited and unlimited)
			\item Defining the Lagrange rest and of Alembert's, Cauchy, integral test and absolute convergence criterias
			\item Developments relative to the definitions of brightness, luminosity, apparent and absolute magnitude of Stars and calculation of the distance to Cepheids
			\item Definitions of the solid angle, the solid angle of revolution and the elementary solid angle
			\item Definition of photometric and photonic and international system quantities
			\item Definitions and developments related to the light intensity, energy flow (with proof of Beer-Lambert' law), emittance, radiance (with Lambert's law), Kirchhoff's law
			\item Proofs of Stefan's law and Stefan-Boltzmann law
			\item Resolutions of third degree polynomials by radicals (Cardan's method) and development about solving quadratic polynomials in the complex set
			\item Definition of the concept of equations and inequalities
			\item Determination of the Cartesian equation of the plane, line (in space),cone and sphere
			\item Definition of market efficiency
			\item Determination of the Black \& Scholes equation 
			\item Presentation of the mathematical aspects of Wiener process
			\item Ito lemma and Brownian motion (random walk)
		\end{itemize}
	\item \textbf{January 12004}
		\begin{itemize}[noitemsep]
			\item Definition of the indefinite integral
			\item Newtonian cosmological model (without the cosmological constant)
			\item Statement of the Peano's
			\item Introduction to Linear Algebra (Gauss reduction method, elementary operations between matrices)
			\item Statement of 5 Euclid's axioms and 5 groups of axioms in geometry
			\item Proof of relations for calculating the perimeter of the surface and the center of gravity of the square, the rectangle and triangle
			\item Proof of relations for calculating the volumes and surfaces of the torus, the sphere, the ellipsoid, cylinder and cone
			\item Definition of the centroid (center of gravity) and demonstration of four properties related thereto
			\item Demonstration of decomposition odd function and pair of any function
			\item Definition of hyperbolic trigonometric functions and enumeration of relations and proof of related remarkable properties
			\item Introduction to differential geometry (definition of a Riemannian geometry, Frenet triad, parametrised surface, etc.)
			\item Introduction to graph theory (Königsberg's bridges proof)
			\item Definition of a topological space and Hausdorff definition of a metric / ultra-metric space and associated distances (hölder, discreet, equivalent ...)
			\item Definition of a Lipschitz function (and related isometrics)
			\item Definition of open and closed set (open/closed balls, spheres, adherence, Hausdorff's excess) and diameters
			\item Definition of set distances (gap) and of a variety / map / atlas and differential homeomorphism
		\end{itemize}
	\item \textbf{April 12004}
		\begin{itemize}[noitemsep]
			\item Proof of the Guldin's theorem
			\item Proof of König's theorem of kinetic energy and angular momentum
			\item Presentation and proofs of various techniques for calculating the moments of inertia: Huygens-Steiner theorem, polar inertia moment, inertia tensor, generalized Huygens-Steiner
			\item Definition of "power" and "performance" and proof of the calculation of the power of a rotating machine
			\item Proof of Boltzmann thermodynamic entropy law and the following statistical distributions: Maxwell speeds, Maxwell-Boltzmann, Fermi-Dirac, Bose-Einstein
			\item Proof of some principal moments of inertia of the following bodies: torus, spheres, cones, rectangular plate, tube
			\item Introduction to wave optics: Huygens principle, proof of Malus law, development of Fraunhofer model in the case of a rectangular slot. 
			\item Definition and proof of the resolving power of a simple rectangular slot.
			\item Proof of the origin and the solution of the no less famous Bessel differential equation of order $n$
			\item Proof of the wave function of a stretched rope and a stretched circular membrane
			\item Proof of Planck's law and of known approximations (first Wien's Law, Rayleigh-Jeans law).
			\item Proof of the displacement law (second Wien's law) and the Stefan-Boltzmann law through Planck's law and determination of the Stefan-Boltzmann constant
			\item Study of the origin of Planck dimensions: Planck length, Planck mass, Planck density, Planck time, Planck energy
		\end{itemize}
	\item \textbf{July 12004}
		\begin{itemize}[noitemsep]
			\item Proof of the physical origin of heat
			\item Proof of Torricelli's theorem
			\item Proof of Venturi effect
			\item Proof of Poiseuille's law
			\item Proof of relativistic Compton effect
			\item Proof of the existence of the fossil radiation in the Universe
			\item Introductions to quotients sets (in this case $\mathbb{Z}/n$
			\item Proof of the physical origin of heat
			\item Proof of Torricelli's theorem
			\item Proof of Venturi's effect
			\item Proof of Poiseuille's law
			\item Proof of relativistic Compton effect
			\item Proof of the existence of the fossil radiation in the Universe
			\item Introductions to quotients sets (in this case $\mathbb{Z}/nZ$)
			\item Proof of Lorentz transformations for speed and acceleration
			\item Determination of the relativistic Lagrangian of a free system
			\item Proof and definition of Ricci theorem, of the covariant derivative, of the Ricci identity, of the Riemann-Christoffel tensor, of the Ricci tensor, of the Ricci scalar, identities of both Bianchi and finally Einstein tensor
			\item Mathematical definition of the capacity and determination of the expression of a  parallel plane capacity
			\item Determination of the electrostatic potential energy
			\item Proof of the value of the field of and electric potential of a straight infinite wire.
			\item Determination of the basic properties of electric dipoles as the rigid dipole moment, the presentation of the induced dipole moment, hydrogen bonds, Van der Waals forces, etc.
			\item Definition of the Curie symmetry principle and Statement of 6 resulting properties
			\item Definition of pseudo vectors
			\item Proof of the relations for the volume of the pyramid, the paraboloid, the tetrahedron, octahedron, cube and parallelepiped
			\item Determination of the magnetic field produced by a toroidal solenoid, a rectilinear solenoid infinity and a current loop
			\item Determination of the magnetic field produced by a magnetic dipole and basic definitions of the properties of magnetic materials
			\item Calculation of the Armorial radius and cyclotron pulsation in a non-relativistic framework
			\item Determination of the Lagrangian of the electromagnetic field and by extension in the non-relativistic approximation of the tensor of the electromagnetic field
			\item Introduction to the calculation of the radiation emitted by an accelerated charge (synchrotron radiation, Lienard-Wiechert's retarded potentials)
			\item Calculation of the values of resistors and capacitors in series
			\item Difference between electrical potential and electromotive potential
			\item Proof of Faraday's law and definition of "self inductance"
			\item Proof of Descartes formulas for the concave and convex spherical surfaces and refracting/not refracting as well as for refractive lenses.
			\item Definition of stigma and proof that the parable is strictly stigmatic
			\item Proof of Descartes formulas for thin lenses and conjugation law
			\item Definition of diopter and explanation of various visual disabilities
		\end{itemize}
	\item \textbf{September 12004}
		\begin{itemize}[noitemsep]
			\item Statement of Mach's principle
			\item Presentation of the photoelectric effect and proof of the physical law governing it.
			\item Proof by example that the light can be seen as both a particle or as a wave
			\item Proof of the theorem of monotone class
			\item Detailed proof of generalized Klein-Gordon equation (relativistic particle in a magnetic field) 
			\item Detailed proof of free Dirac equation with explicit Paulis solutions (particles, antiparticles)
			\item Determination of the radius of the atom using the Rutherford-Coulomb scattering(verbatim: determining the cross section for Rutherford)
			\item Presentation of macroscopic and microscopic interactions of X and gamma-rays with matter (which includes details study of the materialization of a photon in electron-positron pair)
			\item Introduction to spinors
			\item Definition of operating properties of matrices, remarkable matrices, determinants, eigenvectors and eigenvalues
			\item Statements of the postulates of wave quantum physics
			\item Determination of the orbitals of the hydrogen-atom
		\end{itemize}
	\item \textbf{November 12004}
		\begin{itemize}[noitemsep]
			\item New Biographies (Smith, Say, Malthus, Keynes, Walras, Pareto)
			\item Presentation and proof of Noether's theorem
			\item Enumeration of some major physical chemical, astronomical constants
			\item Introduction to the theory of speculation: (predictive expectation of a financial asset)
			\item Introduction to the preference theory (Arrow-Debreu model)
			\item Presentation of solutions of the Black \& Scholes equation and remarks on the delta - Proof of the Call-Put parity equation
			\item Determination of initial stock (optimum) within the framework of the supply chain management
		\end{itemize}
	\item \textbf{January 12005}
		\begin{itemize}[noitemsep]
			\item New Biographies (Penrose, Hawking, Turing, Marx) 
			\item Development of "dual" version of Maxwell equations and proof of the origin of the expression (hypothesis) of magnetic monopoles 
			\item Definition of P, NP and NPC class algorithms
			\item Proof of the fundamental theorem of calculus also named "fundamental theorem of integral and differential calculus" 
			\item Presentation of the cGH cube and the Copenhagen interpretation 
			\item Introduction to the mathematical concepts of artificial neural networks 
			\item Introduction to the mathematical concepts of Genetic Algorithms
			\item Detailed mathematical introduction to fractals
			\item Basics maths stuffs on Quantum Computing
			\item Introduction to fuzzy logic
			\item Proof of Shannon theorem, Morgan theorems, expansion theorem, Karnaugh maps, complete adder, full subtracter
		\end{itemize}
	\item \textbf{April 12005}
		\begin{itemize}[noitemsep]
			\item Basic definitions on block codes, linear codes, systematic codes in error correcting codes
			\item Proof of the relation of the relativistic change in mass
			\item Introduction to codes and prefix codes
			\item Proof of the formula for the calculation of the number of days between two given dates 
			\item Rounding calculations techniques
			\item Definition of rigid or non-rigid post and praenumerando annuities with or without constant rate (certain future)
			\item Definition and study of the properties of loans repayment or constant annuity
		\end{itemize}
	\item \textbf{April 12006}
		\begin{itemize}[noitemsep]
			\item Presentation of Schild's criterion via the Einstein effect (gravitational redshift)
			\item Development of the Newtonian approximation of the geodesic equation
			\item Definitions and proofs of developed forms of four-vectors of displacement, velocity, current, acceleration and energy-momentum
			\item Proof of the provenance of electromagnetic field tensor and referential calculation changes
			\item Definition of a tautology and principle of non-tautology
			\item Descriptions, definitions and many proofs on quaternions
			\item Proof of Euler's number irrationality
			\item Definition of the log-normal, triangular and Weibull distribution and proof of their mean and standard deviation
			\item Introduction to error calculation (absolute and relative uncertainties, error propagation, significant numbers, etc.)
			\item Proof of the deflection of light near a Star with the Newtonian gravitational model
			\item Definition of a rotation matrix (and developments related thereto)
			\item Proof of the existence of the Euclidean division in the ring of polynomials
			\item Definitions of MWRR (Time of Money Weighted Return) and TWRR (Time Weighted Rate of Return)
			\item Proof of Gauss-Ostrogradsky theorem 
		\end{itemize}
	\item \textbf{July 12006}
		\begin{itemize}[noitemsep]
			\item Definition of the concept of dual space
			\item Definition of Pareto law and proof of mean and standard deviation
			\item Definition of quantile (quartile, percentile)
			\item Proof that the mode is the value that minimizes the absolute dispersion
			\item Proof of Minkowski and Bienaymé-Tchebychev inequalities
			\item Proof of the weak law of large numbers 
			\item Proof of Euler's formula for planar graphs
			\item Introduction to the mathematical method of Six Sigma process controls
			\item Proof of the variational calculus theorem
			\item Proof of the calculation of the apparent superluminal speed of a high Redshift Star
			\item Zero-sum game resolution method using operational research
			\item Definition of investment funds
			\item Proof of the beta expression of a simple linear regression model (Sharpe)				
		\end{itemize}
	\item \textbf{October 12006}
		\begin{itemize}[noitemsep]
			\item Proof Binomial and Poisson likelihood estimators
			\item Presentation of the concept of "color" and subtractive and additive synthesis
			\item Proof of Einstein equation fields (approach using weak fields approximation)
			\item Differentiation of the equivalence principle, weak equivalence principle and Einstein's equivalence principle
			\item Proof of the duration of the Daytime arc of planets in the approximation of the null precession and nutation
			\item Digital and formal study (approximative) of Lagrange points of a binary system
			\item Method of resolution of 4\textsuperscript{th} degree polynomial (Ferrari's method)
			\item Presentation of the Gram determinant via the Euclidean volume represented by the joint product of the vectors of a canonical basis
			\item Definition of monotonic, strictly monotone functions, etc .. without pure formal approach
			\item Approximative determination via the Yukawa potential (mass fields) of the mass of mesons of the weak interaction and the strong nuclear interaction.
			\item Introduction to first order linear differential equations			
		\end{itemize}
	\item \textbf{December 12007}
		\begin{itemize}[noitemsep]
			\item New Biographies (Heckman, McFadden, Tesla)
			\item Bernoulli numbers and polynomials
			\item Roche Limit
			\item Flattening of celestial bodies
			\item Pressure and kinetic temperature
			\item Proof of a special case of magnet or electromagnet force
			\item Fundamental introduction to Hermitian and Hilbert vector spaces 
			\item Schwarzschild solution in General Relativity
			\item Precession of the perihelion of Mercury in General Relativity
			\item Light Deflection in General Relativity
			\item Shapiro delay in General Relativity
			\item Evaluation Model of Financial Assets
			\item The Black Hole Universe
			\item Coupling constants of fundamental interactions
			\item Hamiltonian of the Schrödinger equation for a charged particle in an electromagnetic field
			\item Markov chains
			\item Theory of queues
			\item Introduction to weather and marine engineering mathematics
			\item Practical example in Microsoft Excel of the efficiency Markowitz model
			\item Practical example in Microsoft Excel of the Sharp efficiency model
			\item Simplified proof of Green(-Riemann)'s and Stokes theorem
			\item Detailed proofs on principal component analysis (PCA)
			\item Proof of the spectral theorem in the real numbers case
			\item Detailed proofs on logistic regression
			\item Proof of Rolle's theorem and mean value
			\item Proof of the Hopital theorem and generalized finite increments (generalized mean value)
			\item Introduction of Geometric law and proof of its variance and mean
			\item Proof of calculation of the area and volume of the five regular Platonic polyhedra
			\item Introduction to field algebra and geometry
			\item Detailed calculations of the collapsing critical mass (Jeans' Mass) and critical radius (Jeans' Radius) of an interstellar cloud or stellar nurseries
			\item Detailed calculations of the collapse time of an interstellar cloud
			\item Detailed calculations of a Star nuclear life
			\item Detailed mathematical introduction to Fourier transforms
			\item Detailed resolution of homogeneous linear differential equations with constant coefficients
			\item Presentation of the Cornu's spiral for civil engineering
			\item Mathematics of biometric functions
			\item Determination and simple resolution of the Pauli equation
			\item General solution (Fourier transform) of the electromagnetic wave equation
			\item Introduction to strength of materials
			\item Visual Horizon
			\item Growth rate of a population depending on the temperature
			\item Pitot Tube and Pressure drop
			\item $U(1)$ gauge theories in quantum field physics
			\item Bezier curves
			\item Differential equation system with matrix exponentiation
		\end{itemize}
	\item \textbf{September 12009}
		\begin{itemize}[noitemsep]
			\item New Biographies (Pearson, Gosset, Fisher)
			\item New important usual primitives for civil engineering and analytical mechanics
			\item Detailed proofs of the origin of arcsinh and arccosh functions
			\item Proof of Laplace equation and Mayer relation
			\item Proof of the propagation of pressure waves
			\item Historical introduction for nuclear physics
			\item Proof of Maxwell relations in thermodynamics and introduction to free energy and enthalpy
			\item Adiabatic atmosphere model
			\item Proof of Lorenz equations and the butterfly effect
			\item Calculations of some optical properties of the prism
			\item Conditions of decoherence/interferences of electromagnetic waves
			\item Bloch Sphere
			\item Proof of the minimum surface of revolution volume
			\item Treatment of the free particle
			\item Treatment of the polarized spin qubit and the qubit of spin 1/2
			\item Characteristic function and central limit theorem
			\item Some proofs on the inequalities in triangles
			\item Proof of the volume of a barrel with circular section 
			\item Proof of the origin of the mean and standard deviation to the Student and Fisher-Snedecor laws
			\item Introduction to the marginal cost
			\item Statistical test of the one-way ANOVA
			\item Statistical Pearson's Chi-square adjustment test  
			\item Introduction to the analysis of variance of regression
			\item Correspondence Factor Analysis
			\item Development of linear free or forced RC, RL, RLC circuits
		\end{itemize}
	\item \textbf{September 12008}
		\begin{itemize}[noitemsep]
			\item Simple or complex topology systems analysis for preventive maintenance
			\item Introduction to finite and infinite continued fractions
			\item Detailed Solutions of a rectangular tunnelling barrier
			\item Mathematical model of alpha decay via tunnelling
			\item Introduction to Brownian motion model according to Langevin model
			\item Introduction to tribology/friction
			\item Introduction to time series analysis
			\item Additional proofs on long term and short term process capability indices and measuring devices in Statistical Process
			\item Proof of the calculation of the PPM in SPC centered or non-centered process
			\item Proof of Taguchi quality cost relation
			\item Proof of Lienard-Wiechert expression of electric and magnetic potentials 
			\item Introduction to complex analysis
			\item Proof of the second Friedmann equation in cosmology
			\item Proof of "slowing" of light near a Black Hole
			\item Proof of the expression of the Taylor expansion of a function of two real variables
			\item Introduction to experimental design (DoE)
			\item Proof of Eherenfest's theorem
			\item Proof of the likelihood estimators of the Weibull distribution with two parameters
			\item Example of application of decision theory
			\item Bands theory (parabolic and semi-classical approximation) within semiconductors
			\item Theorem of residues and Laurent series
			\item Proofs of Lean Six Sigma values for business processes/workflows
		\end{itemize}
	\item \textbf{October 12011}
		\begin{itemize}[noitemsep]
			\item New Biography (Erlang, Hotelling)
			\item Detailed calculation of the geostationary orbit
			\item Calculations on PVC meteorological probe balloons
			\item Developments on the symmetrical gyroscope and weighing router
			\item Detail calculations and proof on Gini index
			\item Secular balance, transient and non-equilibrium in radioactive filiation
			\item Detailed calculations on the approximate radius of rapidly rotating Stars
			\item Proofs on historical delta-normal and variance-covariance Value At Risk
			\item Two new jokes in the Humor section
			\item Empirical statistical model of wage control
			\item Simplified proof of the possible absence of arbitrage opportunity in Finance
			\item Calculations on self-financing portfolio on the underlying risk
			\item Introduction to mathematical techniques in Insurance
			\item Mathematical developments on the power and intensity of a longitudinal sound wave
			\item Statistical bilateral $Z$-test on the difference in the two means
			\item Statistical Student $T$-test on two paired sample means
			\item Proof of statistical confidence interval relation of sample large proportions
			\item Statistical test for equal proportions of two large samples
			\item Application of Shannon's theorem to calculate a statistical index of diversity
			\item Proof of the determination of coefficients of a multiple linear regression
			\item Proof of the determination of the coefficients of a simple linear regression through the origin
			\item Introduction to sensitivity analysis
			\item Introduction and some proof on rank/order statistics
			\item Demonstration of the provenance, hope and variance of the negative binomial distribution
			\item Control charts with detailed mathematical proofs
			\item Mathematical approach of first Google Page Rank algorithm
			\item Proof of Beltrami's identity to simplify the Euler-Lagrange equation
			\item Exact binomial statistical test for the balance of a population with two characteristics
			\item Developments and study of gravity waves in a fluid
			\item Some simple developments on the gears/gear shafts
			\item Proof of skin's effect
			\item Theory of the rainbow
			\item Theory of double pendulum
			\item Boltzmann distribution law
			\item Dalton's and Amagat laws
			\item Heat Flow
			\item Average power in alternative current
			\item Presentation of some detailed calculations on the betatron
		\end{itemize}
	\item \textbf{May 12013}
		\begin{itemize}[noitemsep]
			\item New Biographies (Napier, Wilcoxon, Born, Heisenberg, Jordan, Kolmogorov, Stokes, Ostrogradsky, Zeeman, Joseph, Faraday, Meitner, Curie, Biot, Debye, Drude, Ohm)
			\item Detailed example of construction of a particular neural network with Microsoft Excel
			\item Resolution of homogeneous linear differential equations of order 1 with non constant coefficients
			\item Example of a Fourier transform of a Gaussian function and proof of the property of the Fourier transform of a derivative
			\item Introduction to interactions in two-factor ANOVA
			\item Confidence interval and prediction interval of a linear regression
			\item Statistical Test signs (median test)
			\item Introduction of conditional probability and the conditional mean of Pareto law
			\item Determination of the estimators of the gamma distribution using the method of moments
			\item Second mathematical approach to the identification of the tides
			\item Statistical Kolmogorov-Smirnov adjustment test with Lilliefors approach
			\item Proof of the Scheafer's quota model
			\item Proof of the calculation of the synodical period of the planets and of demotion of time
			\item Proof of the construction of Brownian bridges
			\item Proof of the origin of the rare events control chart
			\item Calculation of the discount factor of a retiring insurance based on inflation and life expectancy
			\item Proof of two-factor ANOVA relations without repetition and repetition
			\item Mathematical proof and physics of the LASER
			\item Theorem of Taylor series with integral rest
			\item Stability of the Poisson distribution
			\item Statistical Poisson test for 1 or 2 samples
			\item Statistical non-parametric Kruskal-Wallis and Friedman tests
			\item Statistical normality test of Ryan-Joiner
			\item Statistical C Cochran test
			\item Usual Taylor-Maclaurin Series
			\item Polynomial regression by least squares method
			\item Spatiotemporal Finite Difference Method with Maxwell's equations
			\item Break-even Analysis
			\item Mechanical Harmonic Oscillator
			\item Acoustic Doppler Effect
			\item Periodic waves overlays
			\item Statistical Tukey test of range
			\item Forecasting models by moving average, seasonal coefficients, simple smoothing, Brown double smoothing, double Holt (additive) smoothing, double Holt and Winter (multiplicative) smoothing
			\item Introduction to elementary  AR, AM, ARMA and ARIMA autoregressive models
			\item Proof of the expression of the correction factor on finite population
			\item Statistical Fisher exact test
			\item Exponential Laplace Smoothing
			\item Cohen approval Kappa and McNemar statistical test
			\item Kaplan-Meier survival analysis model
			\item Cramer's V
			\item Clustering $K$-Means algorithm
			\item Clustering Dendrograms algorithm
			\item Statistical Wilcoxon signed rank test for one sample or two paired samples
			\item Quantitative study of the effective potential energy (harmonic model of the atomic bonding) of the hydrogen-atom
			\item Proof of beams equation (Euler-Bernoulli equation)
			\item Calculation of the failure rate of a system using the technique of maximum likelihood
			\item We added a chronology of Sciences
			\item Proof of Einstein model (Dulong-Petit law) of the heat capacity of crystalline solids and derivation of the Debye model
			\item Langenvin Model of diamagnetism and paramagnetism
			\item Introduction to line integrals calculation
			\item Naive determination of the energy of a magnetic dipole
			\item "Liquid drop" nuclear Model 
			\item Magnetic model of spin resonance
			\item Integrating factor method for solving differential equations
			\item Constant variation method for solving differential equations
			\item Mendel's law
			\item Temporary and Deferred Life Annuity
			\item Carnot Cycle
			\item Durand and Gordon-Shapiro equity valuation model 
			\item Statistical Anderson-Darling adequation test
			\item Non-linear optimization by the Newton-Quadratic and Gauss-Newton methods
			\item Lagrange Polynomial interpolation method
			\item Statistical Cochran-Mantel Heanzel test
		\end{itemize}
	\item \textbf{November 12016}
		\begin{itemize}[noitemsep]
			\item Singular value decomposition theorem
		\end{itemize}
	\item \textbf{December 12016}
		\begin{itemize}[noitemsep]
			\item Fieller's test (ratio of two means)
		\end{itemize}
	\item \textbf{January 12017}
		\begin{itemize}[noitemsep]
			\item Falling Chimney problem
			\item Levey-Jennings control charts
			\item Simple lattice mixture design of experiment with process variables
			\item Average Failure Rate (reliability)
			\item Markov Chain Reliability Model
			\item Design of reliability tests (Chi-squared time of test, Binomial sampling size, Beta-binomial sampling size)
			\item Weibull distribution linearisation
			\item Inverted pendulum
			\item Durbin-Watson autocorrelation test
			\item Fisher's method for multiple $p$-values
			\item Magnifying glass
			\item Laney's control chart
			\item Classification of conical by the determinant	
			\item Classification of partial differential equations
		\end{itemize}
	\item \textbf{February 12017}
		\begin{itemize}[noitemsep]
			\item Folded Normal distribution basics CDF and PDF
			\item Half-Normal distribution CDF, PDF, variance, mean and median
			\item Telescopic and Gandi's series
			\item Césaro's sum
			\item Implicit Differentiation
			\item Bivariate chain rule
		\end{itemize}
	\item \textbf{March 12017}
		\begin{itemize}[noitemsep]
			\item Laplace Integration method
			\item Lenth's PSE Pareto Margin Error for unreplicated factorial designs
			\item Pareto Margin Error for replicated factorial designs
			\item Design of Experiments desirability
		\end{itemize}
	\item \textbf{April 12017}
		\begin{itemize}[noitemsep]
			\item Friedmann–Lemaître–Robertson–Walker metric
			\item Jensen inequality
		\end{itemize}
	\item \textbf{May 12017}
		\begin{itemize}[noitemsep]
			\item Introducing weak field gravitational waves equation
			\item A mathematical approach of "Divide and rule?" in management
			\item Three new jokes in the Humor section
		\end{itemize}
	\item \textbf{July 12017} (v3.7 $\rightarrow$ v3.8)
		\begin{itemize}[noitemsep]
			\item Correction of many typing errors (almost 50)
			\item Relation 11.3.221 had a integral symbol  (that should not be there)
			\item Relation 7.1.72 had a wrong variable for the angle
			\item Relation 7.5.56 there was a $|a|$ instead of a $|b|$ in the second equality
			\item Relation 11.1.35 had a $4$ that has nothing to do there at the numerator
			\item Relation 15.7.905 there was an alignment issue
			\item Relation 15.7.888 had a $\Lambda$ instead of a $\Gamma$ at the denominator
			\item Relation 11.1.20 had a missing $T$
			\item Relations 11.3.169 must be $r_E$ instead of $r-E$ (LaTeX typing error...)
			\item Table 8.10 and 8.11 had wrong units in the header row
			\item New photo about a quasi vertical circular rainbow and a photo a nuclear magnetic resonance machine
			\item There was a typing error in the period ($T$) relation of the simple pendulum
			\item Details of the back-propagation method for neural network
			\item New mathematical details on the simple linear regression leverage
			\item Numerical applications of some General Relativity experimental tests
			\item Derivation of the geodesic of the sphere (as example in the section of Analytical Mechanics)
			\item ZeroR classification method
			\item K nearest neighbours classification method
			\item Definition of Convexity/Concavity of a function (for Jensen inequality proof)
			\item Proof of  Hermite polynomial Orthogonality
			\item Derivation of the Bachelier Option Pricing Model
			\item Derivation Forward/Future Cox-Ross-Ingersoll Equality
			\item Derivation of the Stress-Energy tensor for a non-relativistic perfect fluid
			\item Derivation of the orthodromic distance
			\item Derivation of the area of a section of an ellipse
			\item There was twice a "Invariance by translation in space" Noether theorem... one of them was obviously "Invariance by translation in time"
			\item Schwarzschild innermost stable orbit
			\item Hafele–Keating experiment with General Relativity treatment
			\item Introducing hyperparameters in Machine Learning
			\item Kernel smoothing
			\item Credit Default risk
			\item We added a lot of new entries in the Chronology section
		\end{itemize}
	\item \textbf{November 12017} (v3.8 $\rightarrow$ v3.9)
		\begin{itemize}[noitemsep]
			\item Section numbering was not correct anymore because of a MiKTeX update.
			\item We created all simple cross references
			\item Multiple errors in PPI (proton-proton Sun chain function reaction) in relations 11.2.27, 11.2.29 and 11.2.30
			\item We added one point to the scientific publication rules (cite equivalent studies for meta-analysis)
			\item Correction of a few hundred minor typing errors
			\item Typing error in the relation 5.4.231, the first line should be a $\cos(x)$ and not a $\sin(x)$
			\item Read "multiplication" instead of "addition" in the remark below the relation 4.2.164.
			\item A square was missing on the $2$ in 4.4.13.
			\item A square was missing on the $r$ in 12.1.57
			\item There was a "$+$" instead of a $\cdot$ in 10.2.21
			\item Square was missing in relation 4.7.1479, 4.7.1480, 4.7.1481, 4.7.1482. 4.7.1484
			\item There was a $c^3$ instead of a $c^2$ in the relation 11.1.212
			\item At 8.3.450 three paragraphs and two equations were duplicates
			\item Proof of the derivative of $f(g)^{g(x)}$
			\item OneR (One Rule) in Data Mining with confusion matrix rule selection
			\item Apriori Association Rule
			\item Kurtosis and Skewness
		\end{itemize}
	\item \textbf{January 12018} (v3.9 $\rightarrow$ v3.10)
		\begin{itemize}[noitemsep]
			\item Typing error in relation 8.2.160 (a $\partial$ missing)
			\item Typing error in relation 9.3.305 (a $\mu$ was missing)
			\item There was an error sign in all inverse Fourier transform (error made during the translation of the book into English).
			\item Dirac continuity equation
			\item CCR (Charnes, Cooper and Rhodes) model of Data Envelopment Analysis (DEA)
			\item Kullback-Leibler divergence
			\item Continuous and Discrete Linear Convolution
			\item Dirichlet Integral
			\item Permutation tests
			\item Binomial and Gaussian naive bayes
			\item More cross-references added
		\end{itemize}
	\item \textbf{June 12018} (v3.10 $\rightarrow$ v3.11)
		\begin{itemize}[noitemsep]
			\item Typing error for the return angle range of the function $\mathrm{atan()}$ ($-$ instead of a $+$)
			\item There was a typing error in relations 4.7.264 and 4.7.265 (power of $3$ instead of $2$).
			\item There was a typing error in relation 4.7.512 a $\text{E}(X)$ instead of a $\text{V}(X)$.
			\item A $c$ was missing in relation 4.5.125
			\item There were typing error in relations 13.1.739 and 13.1.740
			\item A $q$ was missing in relations 9.1.31 and 9.1.32
			\item A $N$ was missing in the third relation 14.4.238
			\item We added a few more Taguchi tables and Taguchi interaction tables
			\item We added the derivation of the one degree of freedom non-central chi-square distribution for the study of tolerance intervals
			\item We added new details about the regression forced through the original (especially on the problem of the $R^2$ calculation)
			\item We added some new images (on data fallacies, a joke with Schrödinger cat, planet's comparison size with respected proportions, an illustration of the difference between bagging and boosting)
			\item We added in the section of Astronomy data tables with the major properties of interest of the planets of our solar system (including Pluto...)
			\item We added in the section of Astrophysics a small data table with the major properties of some well known stars
			\item Almost ten section headers were are at the wrong level in the table of contents (...)
			\item Surprisingly some subjects on Neutron Stars had disappeared (don't ask why...!?). We put them again.
			\item Cox Survival Model (Cox Proportional Hazard Model)
			\item We made more explicit the link between the Gauss-Newton method and the gradient descent method
			\item We added the confidence interval for the Kaplan-Meier estimator and Cox proportional hazard model (i.e. hazard ratio)
		\end{itemize}
	\pagebreak
	\item \textbf{December 12023} (v3.11 $\rightarrow$ v4.0)
		\begin{itemize}[noitemsep]
			\item We noticed that a whole section on Lenz law and inductance wasn't translated since the version 3.1...
			\item Typing error in the equality in the sentence below equation 14.4.125
			\item Translation error in the axiomatic system of Lukasiewicz (the first line had a "Axiom A5" instead of "Axiom A4")
			\item The symbol used for the AND connector in the table 4.5 and the paragraph text were wrong (bad copy paste...)
			\item Wrong copy/paste in relations 13.1.387 and 13.1.388 (the sum symbol and the index $i$ should not have been there)
			\item There was copy/paste error in the relation 7.7.247 (a $\mathrm{d}$ at the beginning that should not have been there)
			\item There were some sign errors in the relations 7.1.20
			\item Two terms were missing in the relation 4.7.356
			\item There were typing errors in relations 8.5.92, 8.5.94 and 8.5.95
			\item There was a copy/paste error in the definition of the weighted average 4.7.39
			\item There was a wrong $2$ factor in the definition of the covariance 4.7.157
			\item We replaced all the $x^{'}$ and $x^{''}$ by $x^\prime$ and $x^{\prime\prime}$
			\item There was a typing error for the energy stored in a capacitor
			\item There was a typing error in table 5.11 for the tautology ($P \vee \neg Q$ instead of $P \vee \neg P$). Thanks to Matvii Hodovankiuk!
			\item Improvements and corrections in the subsection relatively to the relativistic redshift
			\item Improvements and corrections about the SVD theorem
			\item We added a small biography of F. Yates, A. Friedmann, G. Lemaître, G. Cramer
			\item We improved the FRLW cosmological models subsection
			\item We added a few dozens more of spherical harmonics in the section of Quantum Chemistry (thanks to Wikipedia!)
			\item We added a brief introduction to the concept of $XYZ$ analysis
			\item We introduced the concept of ideal mechanical advantage and actual mechanical advantage
			\item Introducing LSTM and GRU recurrent neural networks
			\item We corrected all the misspelling of "well know"...
			\item Three new jokes in the "Humour" section
			\item Mallows's $C_p$ and Akaike Information Criterion (AIC) derivation
			\item LASSO, Ridge and Elastic Net regularize regressions
			\item Cross-over methods including the proof of the CV and GCV
			\item Derivation of multinomial mean and variance
			\item Weighted linear regression and general ordinary least squares
			\item $LU$, $QR$ and Cholesky decomposition details
			\item Gabriel's Horn finite volume and infinite surface solid of revolution
			\item Derivation or Black Hole's photon-sphere radius
			\item Risk Parity portfolio optimization
			\item Law of total variance (iterated law of conditional variance)
			\item Robust estimators, $M$-estimators and $W$-estimators
			\item Hertzian dipole (antenna) in harmonic regime in far-field approximation
			\item Newton-Cotes simple integration methods (Simpson's and Trapezoidal methods)
			\item Runge-Kutta methods for solution to ordinary differential equations
			\item Linear ordering isotonic regression
			\item We added the biography of Jerzy Neyman
			\item Fourier series generative art
			\item Rayleigh scattering
			\item Telegrapher equation
			\item Inductance of a coaxial cable
			\item Hotelling's $T$-square test
			\item Double mass harmonic oscillator
			\item Lawson Criterion for D-T
			\item Determinant of block triangular and diagonal matrices
			\item Woodburry matrix identity and Sherman-Morrison relation
			\item Factor Analysis
			\item Total Unduplicated Reach and Frequency Analysis (TURF)
			\item Equivalence CLAss Transformation (ECLAT)
			\item Welch Test with Welch-Satterhwaite equation
			\item Trimmed means $T$-test
			\item Expectation-Maximization algorithm
			\item Euler's Homogeneous Function Theorem
			\item Eurer-Maclaurin theorem
			\item Planck's 11912 zero point energy radiation law
			\item Heisenberg uncertainty principle zero point energy
			\item Bayes Factor
			\item Abraham-Lorentz formula
			\item Adjusted R-squared
			\item Saha ionization equation
			\item Cosmological Radiowave Background recombination temperature
			\item Casimir Effect
			\item Quantum Cosmology (FLWR Wheeler-Dewitt equation)
			\item Cochran–Armitage test for trend
			\item Generative Neural Networks (GAN)
			\item Point biserial correlation
			\item Sellmeier equation  proof
			\item Torque of a current loop (DC motor)
			\item Fresnel circular aperture and circular obstacle diffraction
			\item Moments generating function and cumulants
			\item Variance stabilizing transformation
			\item Laguerre polynomials
			\item Radial solution or hydrogenoid atom
			\item Thin film interference
			\item Fresnel coefficients
			\item Brewster's angle
			\item Mollweide projection
			\item Variance-Bias tradeoff derivation
			\item Variance-bias tradeoff derivation for OLS
			\item Antithetic, Conditional, Stratified sampling and importance sample Monte Carlo variance reduction techniques
			\item Gauss-Markov theorem detailed derivation
			\item Eddington luminosity and temperature limit (spherical case)
			\item Logistic Kolmogorov-Smirnov statistic
			\item Why gold may be yellow according to special relativity
			\item Betz's law
			\item Mirage equation
			\item Gravitational (non-relativistic) effective potential
			\item Gaussian Wavepacket
			\item Discrete and continuous time Kelly criterion
			\item Added the biography of Maclaurin
			\item Density-Based Spatial Clustering of Applications with Noise (DBSCAN)
			\item Proof of the pricing of a perpetual bond
			\item Pressure of most vessels
		\end{itemize}
	\end{itemize}